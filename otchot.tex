\documentclass[12pt,a4paper]{scrartcl} 
\usepackage[utf8]{inputenc}
\usepackage[english,russian]{babel}
\usepackage{indentfirst}
\usepackage{misccorr}
\usepackage{graphicx}
\usepackage{amsmath}
\begin{document}
{\LARGE \textit{Вариант 5}}\bigskip

{\LARGE \textit{Тема: Написать калькулятор (четыре арифметических операции с возможностью их запоминания) – аналог стандартного калькулятора Windows.}}
			
\section{Ход работы}
\label{sec:exp}

\subsection{Код приложения}
\label{sec:exp:code}
\begin{verbatim}
#include <iostream>
using namespace std;
int main()
{
    int a, b;
    a = 0;
    b = 0;
    char s;

    while (true) 
    {
      cin >> a >> s >> b;
      {
          switch (s)
          {
          case '*':
              cout << a * b;
              break;
          case '/':
              cout << a / b;
              break;
          case '-':
              cout << a - b;
              break;
          case '+':
              cout << a + b;
              break;
          case '1':
              return 0;
          default:
              return 0;
          }
      }
    }
}
\end{verbatim}


\section{Код в работающем состоянии}
\label{sec:picexample}
\begin{figure}[h]
	\centering
	\includegraphics[width=0.2\textwidth]{kod2.jpg}
	\caption{Код}\label{fig:par}
\end{figure}
Работа кода представлена на рис.~\ref{fig:par}.

\section{Библиографические ссылки}

Для изучения «внутренностей» \TeX{} необходимо 
изучить~\cite{andreyolegovich}, а для изучения Get лучше
почитать~\cite{proglib.io}.Чтобы понять как работает калькулятор в си++, нужно обратится к~\cite{youtube}. 

\begin{thebibliography}{9}
\bibitem{andreyolegovich}Изучение \LaTeX{}. https://www.andreyolegovich.ru/PC/LaTeX.phpbase
\bibitem{proglib.io}Изучение Get. https://proglib.io/p/git-for-half-an-hour/    
\bibitem{youtube}Калькулятор. https://www.youtube.com/watch?v=jyDyqaCZvVs
\end{thebibliography}

\end{document}